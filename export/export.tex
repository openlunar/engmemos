\documentclass[12pt]{olfmemo}

\title{An engineer's history of US export controls}
\author{
        Juno~Woods,~Ph.D. \\
        Senior Researcher\\
        Open Lunar Foundation\\
        San Francisco, California
}
\date{\today}
\docnumber{2021-01}
%\date{November~19,~2019}

\usepackage[square,numbers]{natbib}
\usepackage{amsmath}
\usepackage{hyperref}
\bibliographystyle{IEEEtranN}

\begin{document}
\maketitle

\section{Introduction}
It is difficult to measure the effect of export controls on the civil space industry. This set of laws and policies originated during a time when the bulk of space sector work was in the service of US government programs. The overhead imparted by the International Traffic in Arms Regulations (\textsc{itar}) and the Export Administration Regulations (\textsc{ear}) was dwarfed by space shuttle era launch costs. For modern companies like SpaceX, Rocket Labs, and Astra and their clients, regulatory burdens are more significant. For companies with fewer resources, it is often viewed necessary to paint with a broad brush when setting internal policies related to \textsc{itar} and \textsc{ear}, particularly when the hourly rate for an export control attorney is in the ballpark of 800 USD.

The rules that apply to government-funded research differ from those for privately-funded work. Moreover, the governing laws and regulations have changed a number of times over the last two decades, leading to confusion among the engineers who are expected to enforce them. This article seeks to provide a brief history of export control laws in the United States and then discuss some common misconceptions that might be of interest to innovators and entrepreneurs in New Space.

\section{The Cold War and arms control regimes}
In 1940, Congress passed the Export Control Act, the first US peacetime controls on trade (albeit looking down the barrel of World War II). The law included items of strategic military importance as well as commercial items, mirroring the modern distinction between \textsc{itar} and \textsc{ear}. Shortly thereafter, the United States joined the war, and the Act was extended repeatedly and updated periodically. It provided broad authority to the executive branch to set penalties, issue export licenses, and determine the contents of the control lists. Moreover, it exempted the rule-making procedures from most common forms of public and judicial review. It also governed technical data \citep{NAP1987}.

Following World War II, the United States pushed for a multilateral agreement on export controls over the period between 1945 and 1949. Parties to this informal agreement were collectively known as the Coordinating Committee on Export Controls, or CoCom. Like the Export Control Act, CoCom had dual objectives. First, it aimed to strategically prevent equipment for manufacturing armaments from flowing to communist nations, and secondly, it attempted to impose an ``economic `iron curtain''' as described in NSC 68 \citep{NSC68}. This arrangement variably included the US, the UK, France, Belgium, the Netherlands, Denmark, Canada, Luxembourg, and Germany \citep{Yasuhara1991}, though others joined over time. CoCom generally required unanimity to add an item to its control list.

By the sixties, nuclear proliferation was of significant concern. While nuclear bombs were brought to bear in World War II, the invention of space launch technology in 1957 by the Soviet Union (Sputnik) enabled these devices to be delivered ballistically. In the third Nixon--Kennedy presidential debate in 1960, Senator John~F.~Kennedy expressed concern ``that 10, 15, or 20 nations will have a nuclear capacity, including Red China, by the end of the Presidential office in 1964'' \citep{NixonKennedy3rd1960}.

As such, the years 1965--1968 saw the negotiation of the Treaty on the Non-Proliferation of Nuclear Weapons or NPT. The central bargain of the NPT is that non-nuclear-weapon states agree not to acquire nuclear weapons in exchange for the use of peaceful nuclear technology provided by nuclear-armed states. A key theme emerging from this period was the inherent challenge with all dual-use technologies, that their use or misuse depends often on the intent of those possessing the tech. The NPT was the first of three binding treaties on weapons technology, the others being the Biological Weapons Convention of 1972 and the Chemical Weapons Convention of 1993 \citep{Beck2019}, which are outside the scope of this article.

By 1969, the commercial sector had begun to surpass the military in several key technological areas. Under detente, the 1969 revision of the Export Control Act was known as the Export Administration Act (EAA). While the National Security Council recommended bringing the US control list mostly in line with the CoCom list, the Nixon Administration partially blocked the effort. \citep{NAP1987}

In 1974, India tested a `peaceful' nuclear device, resulting in the creation of a second non-binding multilateral export control regime, the Nuclear Suppliers' Group. India had attained a \textsc{candu} nuclear reactor and heavy water from Canada and the United States through President Eisenhower's 1953 ``Atoms for Peace'' program, an attempt to emphasize the peaceful uses of nuclear technology amid concerns about the nuclear arms race \citep{Walker2001}. India, having never signed the NPT, was not bound by the treaty, and a need was seen for a suppler-side agreement to require acceptance of International Atomic Energy Agency safeguards before exporting to any non-nuclear-weapons state \citep{Burr2014}. So-called `full scope' safeguards (on the entire fuel cycle) would not be realized by the NSG until after the Gulf War in the 1990s \citep{Anthony2007}.

The International Traffic in Arms Regulations (ITAR) originated in the 1976 Arms Export Control Act, which gave the executive branch the authority to regulate ``defense articles (arms, ammunition, and implements of war), defense services, and directly related technical data.'' The Department of State was and is responsible for administering these regulations, though the Defense Department generally determines the contents of the control list \citep{NAP1987}, which is known as the US Munitions List, or USML.

The modern Export Administration Regulations (EAR) were created a short time later, by the 1979 revision of the EAA, and are generally said to be more complex than ITAR. It authorized controls on the part of the Commerce Department with several different justifications, all of which fall under the umbrella of dual use technologies. Firstly, national security items were largely drawn from CoCom. Secondly, it permitted controls advancing foreign policy goals. Thirdly, controls might be used to ensure US access to resources. Interestingly, the regulations included a general license for published, scientific, or educational technical data, meaning that no export license was required for these data \citep{NAP1987}. This exemption for information already available to the public remains in place today.

The eighties saw the creation of two new supply side multilateral export control regimes, supplementing CoCom and the Nuclear Suppliers Group. The Australia Group (1985) dealt with chemical weapons, and is not discussed further in this article. The fourth, the Missile Technology Control Regime, was formed in 1987, and is especially impactful upon space technology. Whereas the NSG focused on nuclear technology, the MTCR attended to the delivery technology.

The Missile Technology Control Regime was a Reagan administration response to apprehensions relating to several missile and rocket tests, by South Korea, Iraq, and India, among others \citep{Scheffran1992}. The purpose was to ``reduce the risks of nuclear proliferation by placing controls on equipment and technology transfers which contribute to the development of unmanned, nuclear-weapon delivery services'' \citep{Fialka1987}, though later it expanded to include weapons of mass destruction generally. The MTCR had seven founding members, which has grown to thirty-five today. The USML incorporates all items in the MTCR Annex (according to ITAR, 22 CFR Sec.~120.29).

The end of the Cold War saw the end of CoCom in 1993; CoCom was replaced by the Wassenaar Agreement in 1996. The Wassenaar Agreement largely aimed to serve the same purpose, but with an emphasis on regions of concern rather than the Warsaw Pact. The WA differed from CoCom in several key respects. Firstly, it was significantly larger than CoCom, making consensus difficult. Secondly, it offered greater transparency as compared to CoCom, even publishing a website. Thirdly, it lacked the power to veto exports of controlled items, for which member states previously had to seek out authorization (though neither were CoCom's lists binding). The WA is largely seen today as the weakest and least effective of the major multilateral export control regimes.

\citet{Lipson1999} argued that these multilateral export control regimes relate to shared identity and shared norms between member states. \citet{Joyner2004} called the regimes `security communities,' though \citet{Beck2019} point out that regime members --- unlike those of other security communities --- don't necessarily view the threat of force against one another as unthinkable. \citet{Abbott2000} distinguished between `hard' and `soft' law, arguing that these regimes are soft law, which are less threatening to nations' senses of national security and sovereignty, and more adaptable than treaties.

\section{The evolution of modern US export controls}

A key theme following the Cold War was that of growing technical capabilities of academia and private industry, whereas previously the controlled exports were those of the US government (or often government contractors). Space technologies, particularly communications satellites, bounced back and forth between the USML and the CCL several times over the next two decades. Without commercial rockets, however, inexpensive launches were in high demand among US companies, and this need often caused satellite manufacturers to turn to China.

Yet with the end of the Cold War, US national security concerns shifted from the former Soviet Union to China. \citet{Zinger2015} has provided an excellent history of the policy aspects of export controls from the 1990s through around 2015.

Chinese Long March~2E rockets carrying US satellites --- the Hughes Optus~B2 on 21 December 1992 and the Hughes Apstar~2 on 26~January 1995 --- experienced launch failures. Investigations by Hughes pointed to the fairings as the source of the accidents. According to the 1999 Cox report, confusion over jurisdiction and licensing requirements on the part of both Hughes and government officials led to Hughes relaying technical assistance for improving Long March fairings to China after both failures. In 1995, for example, the fairing --- being a piece of the rocket --- was regulated under ITAR, but a Commerce official, assuming it to be part of the satellite, mistakenly approved the disclosure. \citep{Cox1999}

The nature of the technical data Hughes provided to the PRC after the first failure makes for fascinating reading. Hughes could not obtain insurance for launches subsequent to Optus~B2 without a technical solution. The PRC was unwilling to acknowledge that the fairing was the cause, allegedly for political reasons. The approved but unlicensed transfer included two simple recommendations for changes to the fairing: ``Add a bracket or block to prevent any possibility of overlap of the two fairing halves,'' and ``Increase the strength of the rivets along the separation line'' (which would prevent the fairing from opening prematurely). What Hughes officials viewed as fixes to design deficiencies, the government contended were improvements. \citep{Cox1999}

The Cox Report also contended that Hughes had participated in a collaborative sharing of information with the PRC; each had done an investigation and then shared the results with one another. This type of release 

 and the Loral Intelsat~708 on 15~February 1996 --- experienced rapid deconstruction events. In each case, the companies had sought and received export licenses for the launch. 

and in each case, accident investigation technical data were shared with China.

Shortly after the Intelsat crash, on 14 March, the Clinton administration announced that licensing authority for commercial communication satellite exports would be shifted from from State to Commerce as well, in this case to incentivize China to participate more fully in the MTCR. (The text of the order is available in \citet{State61FR56894_1996}.)



\section{Export controls and the First Amendment}
While a graduate student at the University of California, Berkeley, Daniel Bernstein developed Snuffle, an encryption algorithm. Knowing that the USML regulated some encryption technologies under ITAR, he asked the Department of State if he needed an export license to publish Snuffle, either in source code form or as an academic paper.
\begin{quote}
The State Department responded that Snuffle was a munition under the International Traffic in Arms Regulations...and that Bernstein would need a license to ``export'' the Paper, the Source Code,
or the Instructions. There followed a protrated and unproductive series of letter communications between Bernstein and the goverment, wherein Bernstein unsuccessfully attempted to determine the scope and application of the export regulations to Snuffle. \citep{Bernstein1997}
\end{quote}
%
Bernstein challenged the law in court, arguing that it was a prior restraint on his First Amendment rights to free expression.\citep{Bernstein1997} (Prior restraint is generally considered to be the most abhorrent class of free speech violation, as it seeks to restrain an individual from speaking in the first place rather than to punish speech.)

While Bernstein's challenge wound its way through the courts, the Clinton administration transferred jurisdiction of encryption from State to Commerce \citep{ExecOrder13026_1996}.

%

%Most aerospace engineers are familiar with CoCom in terms of the so-called ``CoCom limits'' in global navigation satellite system (\textsc{gnss}) receivers.


%Perhaps the largest impact of export controls is on hiring. The Defense Department has noted aerospace and defense companies face a skills gap in the native-born population \citep{DoD2018}, yet most US aerospace job postings include a statement that applicants must be United States citizens (or at least US persons). In contrast, foreign-born workers made up nearly one-sixth of the labor force in 2014, and over 70\% of creative information technology roles in Silicon Valley; most were not US citizens \citep{Otoiu2017}.



\bibliography{export}
\end{document}